\documentclass[12pt,a4paper]{report}
\usepackage{graphicx}
\usepackage{amssymb}
\usepackage{amsmath}
\usepackage{mathrsfs}
\begin{document}

%TITLE PAGE
\begin{titlepage}
	
	\begingroup
	\centering
	{\huge\bfseries Elliptic Curve Cryptography\par}
	\vspace{1cm}
	{\scshape\Large Dissertation\par}
	\vspace{1.5cm}
	{\scshape\LARGE Integrated Masters of Science \par}
	In\\
	{\scshape\LARGE Applied Mathematics \par}
	\vspace{2cm}
	Submitted by\\
	{\Large\itshape Harsh Kumar Chourasia\par}
	\vspace{0.5cm}
	supervised by\\
	Dr. Ram Krishna Panday
	\vspace{0.5cm}
		
	\includegraphics[scale=0.75]{logo}
		
	\vspace{1cm}
		
	{\scshape\large Department of Mathematics\par}
	{\scshape\large Indian Institute of Technology, Roorkee\par}
	{\scshape\large Department of Mathematics\par}
	{\scshape\large Roorkee-247667\par}
	\endgroup
\end{titlepage}




%DECLARATION PAGE
\pagenumbering{roman}
\section*{Declaration}
\addcontentsline{toc}{section}{\numberline{}Declaration}
I hereby certify that the work which is being presented in the thesis entitled \textbf{Elliptic Curve Cryptography} in the partial fulfillment of the requirement
for the award of the degree of Integrated Master of Science in Applied Mathematics
and submitted to the Department of Mathematics, Indian Institute of Technology
Roorkee, is an authentic record of my own work carried out during a period from
January 2022 to April 2022 under the supervision of \textbf{Dr. R.K. Panday}, Associate Professor, Mathematics Department, Indian Institute of Technology Roorkee.
The matter presented in this report has not been submitted by me for the
award of any other degree of this or any other institute.\\\\\\
 
\begin{tabular}{l}
Harsh Kumar Chourasia\\
I. M.Sc, Applied Mathematics\\
Department of Mathematics\\
IIT Roorkee\\
Date:\\
Place:\\
\bigskip 
\end{tabular}
\hrule  
\bigskip 
\bigskip 
\noindent{\large{\textbf{CERTIFICATE}}} \\\\
This is certified that the above statement made by the candidate is correct to the best of my knowledge.\\\\\\
\vspace{1cm}
\begin{tabular}{l}
Dr. RK Panday\\
Associate Professor\\
Department of Mathematics\\
IIT Roorkee\\
Date:\\
Place:\\
\end{tabular}
\cleardoublepage	
%ABSTRACT PAGE
\section*{Abstract}
\addcontentsline{toc}{section}{\numberline{}Abstract}
This text discuss about Cryptography using Elliptic Curves. It has many practical applications in end-to-end encryption, data and password storing, cryptocurrencies. The motive behind this dissertation is to understand in depth how cryptography is applied in modern technologies such as blockchain which is driving force behind financial revolution of $21^{st}$ century. This dissertation is divided into two chapters. The first chapter deals with the pre-requisite from abstract algebra and number theory that is required to study elliptic curve cryptography. The second chaptes deals with the introduction and algorithms of elliptic curve cryptography.
\cleardoublepage
%ACKNNOWEDGMENTS PAGE	
\section*{Acknowledgments}
\addcontentsline{toc}{section}{\numberline{}Acknowledgments}
I would like to thank my supervisor, Ram Krishna Panday, Department
of Mathematics, Indian Institute of Technology Roorkee for his guidance through each stage of the process.

I would finally like to thank Prof Premananda Bera, Head of Department, Department of Mathematics, Indian Institute of Technology for giving me the permission to carry out this work at IIT Roorkee\\\\\\
\begin{tabular}{l}
Harsh Kumar Chourasia\\
I. M.Sc, Applied Mathematics\\
Department of Mathematics\\
IIT Roorkee\\
Date:\\
Place:\\ 
\end{tabular}
 
\cleardoublepage	
	
	
\tableofcontents
\thispagestyle{empty}
\cleardoublepage	
\
\setcounter{page}{1}
\pagenumbering{arabic}
	
\chapter{Pre-requisite}
This chapter covers topics of Abstract Algebra and Number Theory that are required for cryptography and elliptic curves.\\
This definition of Group, Ring, Field are as follows
\section{Group}
The set  $G$ is equipped with  single operation $*$ such the the $4$ below properties are satisfied is called a Group.\\
$(1)$ Closure: $\forall x,y \in G, x*y \in G$ \\
$(2)$ Additive identity: $\exists 0 \in G$, such that $ \forall x \in G$, $ 0*x=x*0=x$\\
$(3)$ Associative Property: $ \forall x,y,z \in G$, $(x*y)*z=x*(y*z)$ \\
$(4)$ Inverse: $ \forall x \in S, \exists y \in G$ such that$x*y=0$ where $y$ is known as inverse of $x$ and is denoted by $x^{-1}$
\subsection{Abelian Group}
The set $G$ is equipped with  single operation $*$ such the the $5$ below properties are satisfied is called a Abelian Group.\\
$(1)$ Closure: $\forall x,y \in G, x*y \in G$ \\
$(2)$ Additive identity: $\exists 0 \in G$, such that $ \forall x \in G$, $ 0*x=x*0=x$\\
$(3)$ Associative Property: $ \forall x,y,z \in G$, $(x*y)*z=x*(y*z)$ \\
$(4)$ Commutative Property: $ \forall x,y \in G$, $x*y=y*x$ \\
$(5)$ Inverse: $ \forall x \in G, \exists y \in G$ such that $x*y=0$ where $y$ is known as inverse of $x$ and is denoted by $x^{-1}$	
	
So, a abelian group G is a group with $\forall x,y \in G, x*y=y*x$
\cleardoublepage
	
\section{Ring}
A ring is a set $R$ with two operations $+$ and $*$ which satisfy the below properties\\
$(1)$ It is abelian group under $+$ \\
$(2)$ Closure under $*$:  $x,y \in R \Rightarrow	x*y \in R $  \\
$(3)$ Associative under $*$: $x,y,z \in R \Rightarrow	(x*y)*z=x*(y*z) $\\
$(4)$ Distributive property $x,y,z \in R$
$$x*(y+z)=x*y+x*z$$ $$(x+y)*z=x*z+y*z$$
	
\section{Field}
A Field is a set $F$ with two operations $+$ and $*$ with following properties \\
$(1)$Commutative group under $+$\\
$(2)$Commutative group under $*$\\
$(3)$Distributive property $x,y,z \in F$
$$x*(y+z)=x*y+x*z$$ $$(x+y)*z=x*z+y*z$$
\section{Fermat’s little theorem}
Theorem: Let p be any prime number. For any number a such that $p\nmid a$. Then  
$a^{p-1}\equiv 1\pmod p$\\
Proof: Assume p is a prime number and $p \nmid a$ \\
Every integer is congruent to one of $0,1,2,\cdots,p-1\pmod p$\\
Only focus on non zero congruence classes, because $0 \pmod p$ contains all the multiples of p (and $p \nmid a$).
Focus on $0,1,2,\cdots,p-1$\\
Multiply all of these by a:
$$a,2a,\cdots,(p-1)a$$
Show that this is a rearrangement of $0,1,2,\cdots,p-1$\\
Case 1: None of these are congruent to 0.\\
Suppose $r.a\equiv 0 \pmod p$\\
Then $p\nmid r.a$, but this is impossible since $p\nmid a$ and $r<p$\\
Case 2: These are distinct, no two are congruent to each other.\\
Pick two values $r.a$,$s.a$\\
$$0<r<p$$
$$0<s<p$$
Let's show that $r.a \not\equiv s.a \pmod p$\\
So look at $r.a-s.a=(r-s).a$. As $p\nmid a$, so can $p \mid (r-s)?$\\
$$0<r<p$$
$$-p<-s<0$$
Adding these inequalities gives you:
$$-p<r-s<p$$ 
So, $p\nmid(r-s)$ which means $a,2a,\cdots,(p-1)a$ is a rearrangement of\\ $1,2,\cdots,(p-1).$
$$a,2a,\cdots,(p-1)a\equiv 1,2,\cdots,(p-1) \pmod p$$
$$(p-1)!a^{p-1}\equiv (p-1)! \pmod p$$
$$a^{p-1}\equiv 1 \pmod p$$
		
	
	
\cleardoublepage	

\chapter{Elliptic Curves and Cryptography}
\section{Introduction to Cryptography}
The written word is the most important invention in human history. But as long as human has the ability to share information, they have also had the need to conceal that information as well. This need lead to invention of cryptography.\\
The word cryptography comes from greek which means "hidden writing". According to Wikipedia, \textbf{Cryptography, or cryptology is the practice and study of techniques for secure communication in the presence of adversarial behavior.}
\\Some of the application of Cryptography includes:
\begin{itemize}
	\item End-to-end Encryption for e-mail, messaging apps, GSM phones.
	\item Storing Data: Biggest consumer application of cryptography includes Kindle, iPod which stores books and songs in encrypted format to protect copyright.
	\item Storing Password: Storing passwords in plane text is not secure. If an attacker has access to the system they can read the password. If the password is converted into hash using one way mapping function and stored. Every time a user logs in, the password will be converted into hash and compared with the stored password.
\end{itemize}
There are mainly two type of cryptography: Symmetric key cryptography and Asymmetric key cryptography. 	
\cleardoublepage

	
\subsection{Symmetric cryptography}
Let Alice want to share a message m with Bob. They do so by using a common key and knowledge of some algorithm to encrypt and decrypt message. Alice encrypts the message using the key to produce the cipher text. Now Bob can use key with cipher text to decrypt message. 

In symmetric cryptography a common key is used by the sender and receiver. 
\vspace{2cm}

\begin{figure}[h!]
\begin{center}
 \caption{A picture of the universe!}
\includegraphics[scale=0.36]{sym}
\end{center}
\end{figure}

\vspace{2cm}
\subsection{Asymmetric cryptography}
Asymmetric cryptography works by using private and public key pairs.
Each user has a private,public key pair. Public key can be shared freely across the network and is used to verify the owner of a message.
Private keys is not transmitted across the network. Public are used to encrypt the message and private key is used to decrypt the message. The major advantage of asymmetric cryptography is that there is no need of a shared key. \\

\begin{figure}[h!]
\begin{center}
\caption{A message exchange using private and public keys}
\includegraphics[scale=0.36]{asym}
\end{center}
\end{figure}

	
\section{Elliptic Curves}
Equation of type $ y^2 = x^3+ax+b $ are called Weierstrass equations. It is named after Karl Weierstrass $(1815-1897)$ who studied them in $19^{th}$ century. \\
Definition:Elliptic curves are solution sets of Weierstrass equations
$$E:y^2 = x^3+ax+b ...(1)$$ with  $ \{\ \mathscr{O}  $ \}\ where $\Delta_E = 4a^3+27b^2\neq 0$. 
$\Delta_E \neq 0$ guarantees that the equation $x^3+ax+b$ has no repeated roots i.e. 
$ x^3+ax+b=(x-e_1)(x-e_2)(x-e_3)$ where $e_1,e_2,e_3$ are distinct. $  \mathscr{O}  $  is defined as the point at infinity which lies on every vertical line.\\
\begin{figure}[h!]
\caption{Example of Elliptic Curves}
\includegraphics[scale=0.32]{Figure_1}
\includegraphics[scale=0.32]{Figure_2}\\
\includegraphics[scale=0.32]{Figure_3}
\includegraphics[scale=0.32]{Figure_4}
\end{figure}
\cleardoublepage
If (x,y) satisfies eq(1), then (x,-y) is also a solution of equation (1). So, elliptic curves are symmetric about x-axis. \\
The definition of addition "$+$" operator is a not the usual definition one might expect 
$$(a,b)+(c,d) \neq (a+c,b+d)$$
Two points $P_1$ and $P_2$ on elliptic curve. If we make a line L that passes through $P_1$ and $P_2$, it will intersect the curve at point $P_3=(x_3,y_3)$. The reflection of $P_3$ from x-axis i.e. $(x_3,-y_3)$ is called the sum of points $P_1$ and $P_2$
\begin{figure}[h!]
\begin{center}
\caption{Example of $P_1+P_2$}
\includegraphics[scale=0.4]{2}
\end{center}
\end{figure}
\cleardoublepage
So, what is $P_1+P_1$? This is the limiting case where $P_2 \to P_1$ and the Line L becomes the tangent to E at $P_1$. This line will intersect E at $P_3$. The reflection of $P_3$ about x-axis  is $P_1+P_1$.
\begin{figure}[h!]
\caption{Example of $P_1+P_1$}
\begin{center}
\includegraphics[scale=0.3]{3}
\end{center}
\end{figure}
Let's look at the case when two points on the curve when $P_1=(x,y)$ and $P_2=(x,-y)$ are added. In this case line L is $x=a$. L will not intersect the curve at third point. In this case we define 
$ P_1+P_2= \mathscr{O} $. We define $\mathscr{O}$  as the point in infinity that lies on every vertical line. 
If P = (x,y) then -P is defined as (x,-y). So, $P+(-P)=\mathscr{O}$...(2)
\begin{figure}[h!]
\begin{center}
\caption{Example of $P_1+(-P_1)=\mathscr{O}$}
\includegraphics[scale=0.3]{1}
\end{center}
\end{figure}
\cleardoublepage
Theorem: Let E be Elliptic curve. Then $E$ forms abelian group under addition. The following below statements are true:
 \begin{enumerate}
 \item $P_1+\mathscr{O}=\mathscr{O}+P_1=P_1$ for all $P_1 \in E$
 \item $P_1+(-P_1)=\mathscr{O}$ for all $P_1 \in E$
 \item $(P_1+P_2)+P_3=P_1+(P_2+P_3)$ for all $P_1,P_2,P_3 \in E$
 \item $P_1+P_2=P_2+P_1$ for all $P_1,P_2 \in E$
 \end{enumerate}
Proof:
\\(1) Claim: $P_1+\mathscr{O}=P_1$ \\
If a line is drawn through P and $\mathscr{O}$ it will intersect E at -P. Reflection of -P from x-axis is again P. So, $P_1+\mathscr{O}=P_1$ 
Similarly, $\mathscr{O}+P_1=P_1$\\
(2) Explained above in equation (2)\\
(4) is true as line passing through $P_1$ and $P_2$ is same as the line passing through $P_2$ and $P_1$. 
(3) will be proved using the next theorem.
\cleardoublepage
YET TO EXPLORE PART
 \begin{enumerate}
 \item  Elliptic curves over finite fields
 \item  The elliptic curve discrete logarithm
problem (ECDLP)
 \item  Elliptic Diffie–Hellman key exchange
 \item Lenstra’s elliptic curve factorization
algorithm
 \end{enumerate}
\bibliographystyle{plain}
\bibliography{bibliography}
\end{document}

\documentclass[12pt,a4paper]{report}
\usepackage{graphicx}
\usepackage{amssymb}
\usepackage{amsmath}

\begin{document}

%TITLE PAGE
\begin{titlepage}

	\begingroup
	\centering
	{\huge\bfseries Elliptic Curve Cryptography\par}
	\vspace{1cm}
	{\scshape\Large Dissertation\par}
	\vspace{1.5cm}
	{\scshape\LARGE Integrated Masters of Science \par}
	In\\
	{\scshape\LARGE Applied Mathematics \par}
	\vspace{2cm}
	Submitted by\\
	{\Large\itshape Harsh Kumar Chourasia\par}
	\vspace{0.5cm}
	supervised by\\
	Dr. Ram Krishna Panday
	\vspace{0.5cm}
	
	\includegraphics[scale=0.75]{logo}
	
	\vspace{1cm}
	
	{\scshape\large Department of Mathematics\par}
	{\scshape\large Indian Institute of Technology, Roorkee\par}
	{\scshape\large Department of Mathematics\par}
	{\scshape\large Roorkee-247667\par}
	\endgroup
\end{titlepage}




%DECLARATION PAGE
	\pagenumbering{roman}
	\section*{Declaration}
	\addcontentsline{toc}{section}{\numberline{}Declaration}
	I hereby certify that the work which is being presented in the thesis entitled \textbf{Elliptic Curve Cryptography} in the partial fulfillment of the requirement
for the award of the degree of Integrated Master of Science in Applied Mathematics
and submitted to the Department of Mathematics, Indian Institute of Technology
Roorkee, is an authentic record of my own work carried out during a period from
January 2022 to April 2022 under the supervision of \textbf{Dr. R.K. Panday}, Associate Professor, Mathematics Department, Indian Institute of Technology Roorkee.
The matter presented in this report has not been submitted by me for the
award of any other degree of this or any other institute.


	\cleardoublepage	




%ABSTRACT PAGE
	\section*{Abstract}
	\addcontentsline{toc}{section}{\numberline{}Abstract}
	\cleardoublepage
%ACKNNOWEDGMENTS PAGE	
	\section*{Acknowledgments}
	\addcontentsline{toc}{section}{\numberline{}Acknowledgments}
	I would like to thank my supervisor, Ram Krishna Panday,\\Department
of Mathematics, Indian Institute of Technology Roorkee for his guidance through each stage of the process. 
	\cleardoublepage	
	
	
	\tableofcontents
	\thispagestyle{empty}
	\cleardoublepage	
	\
	\setcounter{page}{1}
	\pagenumbering{arabic}
	
	\chapter{Pre-requisite}
	This chapter covers topics of Abstact Algebra and Number Theory that are required for cryprography and elliptic curves.\\
	 This defination of Group, Ring, Field are as follows
	\section{Group}
	The set  $G$ is equipped with  single operation $*$ such the the $4$ below properties are satisfied is called a Group.\\
	$(1)$ Closure: $\forall x,y \in G, x*y \in G$ \\
	$(2)$ Additive identity: $\exists 0 \in G$, such that $ \forall x \in G$, $ 0*x=x*0=x$\\
	$(3)$ Associative Property: $ \forall x,y,z \in G$, $(x*y)*z=x*(y*z)$ \\
	$(4)$ Inverse: $ \forall x \in S, \exists y \in G$ such that$x*y=0$ where $y$ is known as inverse of $x$ and is denoted by $x^{-1}$
	\subsection{Abelian Group}
	The set $G$ is equipped with  single operation $*$ such the the $5$ below properties are satisfied is called a Abelian Group.\\
	$(1)$ Closure: $\forall x,y \in G, x*y \in G$ \\
	$(2)$ Additive identity: $\exists 0 \in G$, such that $ \forall x \in G$, $ 0*x=x*0=x$\\
	$(3)$ Associative Property: $ \forall x,y,z \in G$, $(x*y)*z=x*(y*z)$ \\
	$(4)$ Commutatitve Property: $ \forall x,y \in G$, $x*y=y*x$ \\
	$(5)$ Inverse: $ \forall x \in G, \exists y \in G$ such that $x*y=0$ where $y$ is known as inverse of $x$ and is denoted by $x^{-1}$	
	
	So, a abelian group G is a group with $\forall x,y \in G, x*y=y*x$
	\cleardoublepage
	
	\section{Ring}
	A ring is a set $R$ with two operations $+$ and $*$ which satisfy the below properties\\
	$(1)$ It is abelian group under $+$ \\
	$(2)$ Closure under $*$:  $x,y \in R \Rightarrow	x*y \in R $  \\
	$(3)$ Associative under $*$: $x,y,z \in R \Rightarrow	(x*y)*z=x*(y*z) $\\
	$(4)$ Distributive property $x,y,z \in R$
	$$x*(y+z)=x*y+x*z$$ $$(x+y)*z=x*z+y*z$$
	
	\section{Field}
	A Field is a set $F$ with two operations $+$ and $*$ with following properties \\
	$(1)$Commutative group under $+$\\
	$(2)$Commutative group under $*$\\
	$(3)$Distributive property $x,y,z \in F$
	$$x*(y+z)=x*y+x*z$$ $$(x+y)*z=x*z+y*z$$\\
	
	\section{Fermat’s little theorem}
	Theorem: Let p be any prime number. For any number a such that $p\nmid a$. Then  
$a^{p-1}\equiv 1\pmod p$\\\\
	Proof: Assume p is a prime number and $p \nmid a$ \\
	Every integer is congruent to one of $0,1,2,\cdots,p-1\pmod p$\\
	Only focus on non zero congurence classes, because $0 \pmod p$ contains all the multiples of p (and $p \nmid a$).
	Focus on $0,1,2,\cdots,p-1$\\
	Multiply all of these by a:
	$$a,2a,\cdots,(p-1)a$$
	Show that this is a rearrangement of $0,1,2,\cdots,p-1$\\
	Case 1: None of these are congruent to 0.\\
	Suppose $r.a\equiv 0 \pmod p$\\
	Then $p\nmid r.a$, but this is impossible since $p\nmid a$ and $r<p$\\
	Case 2: These are distinct, no two are congruent to each other.\\
	Pick two values $r.a$,$s.a$\\
	$$0<r<p$$
	$$0<s<p$$
	Let's show that $r.a \not\equiv s.a \pmod p$\\
	So look at $r.a-s.a=(r-s).a$. As $p\nmid a$, so can $p \mid (r-s)?$\\
	$$0<r<p$$
	$$-p<-s<0$$
	Adding these inequalities gives you:
	$$-p<r-s<p$$ 
	So, $p\nmid(r-s)$ which means $a,2a,\cdots,(p-1)a$ is a rearranement of\\ $1,2,\cdots,(p-1).$
		$$a,2a,\cdots,(p-1)a\equiv 1,2,\cdots,(p-1) \pmod p$$
		$$(p-1)!a^{p-1}\equiv (p-1)! \pmod p$$
		$$a^{p-1}\equiv 1 \pmod p$$
		
	
	
	\cleardoublepage	

	\chapter{Elliptic Curves and Cryptography}
	\section{Introduction to Cryptography}
	According to Wikipedia, \textbf{Cryptography, or cryptology is the practice and study of techniques for secure communication in the presence of adversarial behavior.}
	\\Some of the application of Cryptography includes:
	\begin{itemize}
  \item End-to-end Encryption for e-mail, messaging apps, GSM phones.
  \item Storing Data: Biggest consumer application of cryptography includes Kindle, iPod which stores books and songs in encrypted format to protect copyright.
  \item Storing Password: Storing passwords in plane text is not secure. If an attacker has access to the system they can read the password. If the password is conveted into hash using one way mapping function and stored. Everytime a user logs in, the password will be converted into hash and compared with the stored password.
\end{itemize}
There are mainly two type of crpytography: Symmeteric key cryptography and Asymmeteric key crpytography. 	
\cleardoublepage

	
	\subsection{Symmetric cryptography}
	Let Alice want to share a message m with Bob. They do so by using a common key and knowledge of some algorithm to encrypt and decrypt message. Alice encrypts the message using the key to produce the cipher text. Now Bob can use key with cipher text to decrypt message.  \\\\
	\includegraphics[scale=0.4]{sym}
	\subsection{Asymmetric cryptography}
	Asymetric cryptography works by using private and public key pairs.
	Each user has a private,public key pair. Public key can be shared freely across the network and is used to verify the owner of a message.
	Private keys is not transmitted across the network. Private are used to encrypt the message. The major advantage of asymmeteric cryptography is that there is no need of a shared key. 
	\cleardoublepage	
	\section{Elliptic Curves}
Equation of type $ y^2 = x^3+ax+b $ are called Weierstrass equations. It is named after Karl Weierstrass $(1815-1897)$ who studied them in $19^{th}$ century. \\
Defination:Elliptic curves are solution sets of Weierstrass equations$$E:y^2 = x^3+ax+b$$ where $\Delta_E = 4a^3+27b^2\neq 0$ \\
$\Delta_E \neq 0$ gurantees that the equation $x^3+ax+b$ has no repeated roots i.e. 

$ x^3+ax+b=(x-e_1)(x-e_2)(x-e_3)$ where $e_1,e_2,e_3$ are distinct

Example:

$y^2 = x^3-3x+3$ over $\mathbb{R}$

\includegraphics[scale=0.3]{graph_ex1}

E: $y^2 = x^3+3x+8$ over $\mathbb{Z}/13\mathbb{Z}$\\
Solution is  $O,(1,5),(1,8),(2,3),(2,10),(9,6),(9,7),(12,2),(12,11)$ 




	\section{Elliptic Curves over Finite Field}
	\section{Discrete Logarithm problem}
	\subsection{The elliptic curve discrete logarithm \\ problem}
	\subsection{Double and add algorithm}
	\subsection{How hard is the ECDLP?}
	\section{Elliptic curve cryptography}
	\section{Diffie–Hellman}
	\subsection{Diffie–Hellman key exchange}
	\cleardoublepage
	

	\cite{hoffstein2008introduction}
	
	
	\bibliographystyle{plain}
	\bibliography{bibliography}
	 
	

\end{document}
